\documentclass[	11pt, ]{fphw}
\usepackage[utf8]{inputenc} 
\usepackage[T1]{fontenc}
\usepackage{mathpazo}\usepackage{graphicx} 
\usepackage{booktabs} 
\usepackage{listings} 
\usepackage{amsmath}
\usepackage{eso-pic}
\usepackage{array}
\usepackage{float}
\usepackage{graphicx}
\usepackage{bbm}
\usepackage{transparent}
\usepackage{indentfirst}
\usepackage{amsmath,amssymb}
\usepackage{subfigure}
\usepackage{tabu}
\usepackage{enumitem}
\usepackage{caption,tabularx,booktabs}
\usepackage{enumerate} 
\usepackage[english]{varioref}
\renewcommand{\thesection}{\Roman{section}}
\usepackage{titlesec}
\titleformat{\section}
{\normalfont\Large\bfseries}{Exercise~\thesection}{1em}{}
\makeatletter
\renewcommand{\@seccntformat}[1]{
  \csname the#1\endcsname
  \csname suffix@#1\endcsname 
  \quad
}
\renewcommand{\thesubsection}{\alph{subsection}}
\renewcommand{\p@subsection}{\thesubsection.}
% define \suffix@subsection
\newcommand{\suffix@subsection}{)}
\makeatother

\title{Assignment \#8} %
\author{Anita Mezzetti} 
\institute{École polytechnique fédérale de Lausanne} 
\class{Global Business Environment} 
\professor{Luisa Lambertini} 

%--------------------------------------------------------------------
\begin{document}
\maketitle


\section{}
\subsection{}
True. 
\par 
The risk of default on Mexican public debt us 
\[12\%-5=7\%. \]
We can simply subtract them because they are in the same currency (USD). \\
We can mention the next relation 
\[ R_{cetes}=R{tesobonds}^{*}+\frac{E^{e}}{E}-1 \]
and calculate 
\[ \frac{E^{e}}{E}-1=R_{cetes}-R{tesobonds}^{*}=35\%-12\%=23\% .\]
So the expected depreciation is bigger than the risk premium.




\subsection{}
True. 
\par 
"Modern economic theory says that inflation expectations are an important determinant of actual inflation. Firms and households take into account the expected rate of inflation when making economic decisions, such as wage contract negotiations or firms’ pricing decisions. All of these decisions, in turn, feed into the actual rate of increase in prices. Given that central banks are concerned with price stability, policymakers pay attention to inflation expectations in addition to actual inflation" \footnote{\url{https://www.stlouisfed.org/publications/regional-economist/april-2016/inflation-expectations-are-important-to-central-bankers-too}}. 
So, it is reasonable to talk about actual inflation and inflationary expectation. 
\par Before 2013, "The BOJ has talked about targeting inflation for years without any success, but these changes are more credible" \footnote{ \url{https://www.bbc.com/news/business-21136866}}. 
Bank of Japan engaged in quantitative easing, printing a lot of new money. The goal was to help weaken the value of yen to help exports (Japanese products cheaper in international market). The central bank's plan to pump money into the economy to put back inflation back into the Japanese economy: setting inflation target to 2\%. They wanted inflation because Japan has suffered for many years of deflation, which hold back investments. BOJ hopes that, putting a inflation target, people would start to spend and company to invest. 






\end{document}