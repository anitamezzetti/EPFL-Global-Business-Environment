\documentclass[	11pt, ]{fphw}
\usepackage[utf8]{inputenc} 
\usepackage[T1]{fontenc}
\usepackage{mathpazo}\usepackage{graphicx} 
\usepackage{booktabs} 
\usepackage{listings} 
\usepackage{amsmath}
\usepackage{eso-pic}
\usepackage{array}
\usepackage{float}
\usepackage{graphicx}
\usepackage{bbm}
\usepackage{transparent}
\usepackage{indentfirst}
\usepackage{amsmath,amssymb}
\usepackage{subfigure}
\usepackage{tabu}
\usepackage{enumitem}
\usepackage{caption,tabularx,booktabs}
\usepackage{enumerate} 
\usepackage[english]{varioref}
\renewcommand{\thesection}{\Roman{section}}
\usepackage{titlesec}
\titleformat{\section}
{\normalfont\Large\bfseries}{Exercise~\thesection}{1em}{}
\makeatletter
\renewcommand{\@seccntformat}[1]{
  \csname the#1\endcsname
  \csname suffix@#1\endcsname 
  \quad
}
\renewcommand{\thesubsection}{\alph{subsection}}
\renewcommand{\p@subsection}{\thesubsection.}
% define \suffix@subsection
\newcommand{\suffix@subsection}{)}
\makeatother

\title{Assignment \#7} %
\author{Anita Mezzetti} 
\institute{École polytechnique fédérale de Lausanne} 
\class{Global Business Environment} 
\professor{Luisa Lambertini} 

%--------------------------------------------------------------------
\begin{document}
\maketitle

\section{}\subsection{}
False.
\par If the house prices rises in Canada relative to the US, the Canadian currency increases its value. So the exchange rate E$_{CAD/US}$ decreases and E$_{US/CAD}$ increases. Therefore there is a depreciation of US dollar. If everything else does not cange, we can conclude that


\subsection{}
True.
\par The central Bank of Russia raised the benchmark interest rate, i.e. it raises the price per unit of quantity of a commodity traded in the international marketplace. This leads to a decrease in the money supply: if R grows, L(R,Y) decreases and M$^{s}$ consequently drops too. If the money supply of a country decreases permanently, its long run exchange rate falls: E\up{e} appreciates. Therefore it is reasonable to conclude that the rouble-US dollar exchange rate reached a record high after the rate rise. \\
The Ukraine crisis started at the end of 2013 and it is probably one of the reasons why Russian interest rate increased. This crises, not only contracted the Ukrainian economy, but also led to economic sanctions for Russia. For these sanctions, imposed by western countries, Russian rate of inflation was quite high and foreign products became relatively more expensive. 
\\
In addition, In Russia oil prices falls in 2014. During the second half of 2014, also Americans celebrated a rapid decline in the price of oil and gas. However, we have to consider that some countries prosper when oil prices decline and suffer economically when they rise, while the opposite is true for others. Countries whose economies benefit when the price of oil is low tend to be net importers of oil, so US belongs to these category. The price of oil and Russia's economy have the opposite relationship: when oil prices drop, Russia suffers greatly; considering that oil represents the bigger part of Russian GDP. So the Russian rouble declined in value relative to the U.S. dollar.




\end{document}